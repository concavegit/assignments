\documentclass{tufte-handout}

\usepackage{../CommonLatexPackages/machine_learning_preamble_1.0}
\fancypagestyle{firstpage}

{\rhead{Notation Conventions \linebreak \textit{Version: \today}}}

\title{Notation Conventions}
\author{Machine Learning}
\date{Fall 2019}

\begin{document}

\maketitle
\thispagestyle{firstpage}

\section{Notation}

\subsection{Scalars}
We will use lower-case, unbolded letters to refer to scalar quantities.  For example, we would refer to the scalar quantity x using the following notation.
\begin{align}
x
\end{align}

\subsection{Vectors}

We will use lower-case, bolded letters to refer to vector quantities.  For example, we would refer to the vector quantity v using the following notation.

\begin{align}
\mlvec{v}
\end{align}

\subsection{Matrices}

We will use upper-case, bolded letters to refer to matrix quantities.  For example, we would refer to the matrix quantity A using the following notation.

\begin{align}
\mlmat{A}
\end{align}

\subsection{Independent versus Dependent variables}
We will use `x' to refer to independent (i.e., input) variables and `y' to refer to dependent (i.e., output) variables.  For instance, when describing training data, we will always use `x' to refer to the input variables and `y' to refer to the output variable.


\end{document}
